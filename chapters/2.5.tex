\subsection{Cross-site request forgery}
Cross-site request forgery (CSRF) is an attack where an attacker causes a victim's browser to perform an unwanted operation on a trusted website.
CSRF attacks rely on the fact that stored user sessions belonging to a specific domain tend to be included automatically in all HTTP requests to the domain.
As such, if the victim is tricked to click a link to the target site with some malicious payload, such as a form submission to update a password, the victim's browser will automatically include the victim's session in the request, allowing the attacker to perform operations as the victim.
A typical CSRF attack has the goal of gaining access to resources belonging to the victim \citep{lin_threat_2009}.
This can give attackers access to private resources belonging to the victim or allow attackers to impersonate the victim.
By forging a request that signs the victim in as the attacker to a legitimate site, a victim can be tricked to perform operations as the attacker in \citep{barth_robust_2008}.
These login CSRF attacks are more difficult to defend against than traditional CSRF attacks but can be equally damaging.
An attacker could for example trick the victim to store a payment method in the attackers account.

CSRF attacks can be performed using a number of methods that all rely on somehow causing the victim's browser to visit a link provided by the attacker.
In an ideal case for the attacker, the vulnerable route is a HTTP form that allows the attacker to perform an operation as the victim, such as transferring money or changing account passwords.
On a website with user-generated content, such as a forum, attackers could embed malicious links in images.
For malicious images to be executed on page load the site will need to have a vulnerable route where a HTTP GET request executes an operation with side-effects.
According to the HTTP specification, GET requests should be free of side effects, but it is up to developers to correctly implement the specification.
If a victim visits a malicious site controlled by the attacker, the malicious site can instruct the user's browser to execute any HTTP request to the vulnerable site.
Compared to the malicious embedded images, the target site does not need to have any wrongly configured routes that execute GET requests with side-effects, as the attacker can freely choose the type of HTTP request.

As the potential for CSRF attacks have been known for many years, a number of well-known defenses exist.
Bart et al. (2008) suggest protecting against CSRF attacks by using secret validation tokens \citep{barth_robust_2008}.
A secret validation token is a random value that is included by applications in each HTTP request to ensure that requests have been sent from an authorized source.
The token should be hard to guess for attackers that do not already have access to the victim's account.
As the token will have to be stored in the client, end users will be able to access it with ease in many cases.
As such, the token should never be shared between user or sessions, and should ideally be treated as a single-use nonce to make replay attacks less likely.
