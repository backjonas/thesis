\subsection{Identifiers}
The OAuth specification utilizes a number of tokens and other identifiers that are used to identify resource owners and clients.
\begin{itemize}
    \item \textbf{Authorization code} \\
    The authorization code is granted by an authorization server to a client after the server has successfully authenticated the resource owner.
    The authorization code is a random string, and it does not typically contain any data in itself.
    The authorization code is used by the client to request an access token from the authorization server.
    As the authorization code should only be used once to request an access token, it must only be valid for one request and must be short-lived.
    If an authorization code is used multiple times, the authorization server should revoke all tokens that have been generated using the code.
    In the implicit code grant, an authorization code is not used, with the authorization server issuing an access token directly instead.
    \item \textbf{Access token} \\
    Access tokens are strings which are used to grant access for a specific client to a specific resource server.
    The token additionally includes a scope for the granted access, making it possible to grant granular access to resources.
    To mitigate the chance for leaked tokens being abused, the token can be given an expiration time.
    Access tokens and related attributes have to be kept confidential in transit and storage.
    Access tokens should only be shared between the client it is issued to, the authorization server that creates the token and the resource server which the token is valid for.
    It must not be possible for unauthorized parties to generate, modify or guess valid access tokens.
    \item \textbf{Refresh token} \\
    Refresh tokens are strings which are used to request additional access tokens.
    Refresh tokens are created simultaneously with access tokens, and are useful when the access token has an expiry time set, allowing for end-users to stay authenticated for a longer period of time if the client stays active by refreshing new access tokens before the previous token expires.
    The refresh logic is typically performed in the background by the client, causing minimal interruption for end-users.
    When granting a new access token from a refresh token, the authorization server can issue a new refresh token, allowing for multiple refreshes for one authorization.
    Refresh tokens and related attributes have to be kept confidential in transit and storage.
    Refresh tokens should only be shared between the client it is issued to, the authorization server that creates the token and the resource server which the token is valid for.    
\end{itemize}

