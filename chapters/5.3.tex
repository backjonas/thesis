\subsection{Proof Key for Code Exchange}
\label{sec:pkce}
To protect against authorization code injection, the OAuth 2.0 Security Best Current Practice \citep{ietf-oauth-security-topics-27} recommends using Proof Key for Code Exchange (PKCE) to bind authorization codes to client instances \citep{rfc7636}.

PKCE relies on creating a code verifier in the initial authorization code request (step 2 in figure \ref{client_auth_requests}).
The code verifier should be a random string with between 43 and 128 characters.
This code verifier should be temporarily stored by the client.
In our example client, the value is stored as a cookie in the user's browser.
After generating the code verifier, the client generates a code challenge by hashing the code verifier.
The code challenge should be generated using the SHA256 algorithm \citep{dang_secure_2015} if possible.
The generated code challenge is sent to the authorization server together with the code request.
The used code challenge method, such as \textit{S256} if SHA256 is used, is also included in the request.

After receiving the authorization code, the OAuth client requests an access token from the authorization server (step 6 in figure \ref{client_auth_requests}).
The original code verifier is included in the access token request.
The authorization server computes a code challenge based on the verifier and the previously sent challenge method and compares it with the initial code challenge.
If the two code challenge values do not match, the authorization server interrupts the access token generation.



